\documentclass[11pt]{article}
\usepackage[margin=1.2in]{geometry}
\usepackage[backend=biber]{biblatex}
\addbibresource{references.bib}

\begin{document}

\setlength{\parindent}{0em}
\setlength{\parskip}{1em}
\noindent
Dear Selection Committee,

\vspace{5mm}

I am writing to apply for the 2021 Summer UTEA. I have completed three years of the Computer Science Specialist program and am currently doing a PEY placement at SOTI. My primary research interests are heterogeneous computing and compiler design and development, which I hope to explore in an internship. Among the available projects, I find ``Horizontally Fused Training Array" with Professor Pekhimenko, ``Dependently-typed functional programming for machine learning" with Professor Duvenaud, ``Enabling efficient disaggregated servers in data centers" with Nandita Vijaykumar particularly intriguing.

My computer science education has given me a solid theoretical understanding of computer architecture and concurrency, particularly in a heterogeneous setting. I am passionate about applying this background to practical problems in systems development and design. I have extensive experience developing high-performance low-level software, including highly-optimized multithreaded code, operating systems components, distributed and fault-tolerant networked systems, compilers, and GPU kernels. I also have an interest in scientific computing and its applications, especially in machine learning. Recently, I have participated in the development of RAIN, an RVSDG-based research compiler built on top of LLVM. I was responsible for writing the code-generation module in the backend, which was to generate LLVM IR from the compiler's internal data structures. In the field of machine learning, I have applied convolutional neural networks to the classification of medical images as part of the Image-Guided Radiation Therapy Net project at the Fields Institute.

The project "Horizontally Fused Training Array" interests me as it lies at the intersection of machine learning, compiler development, and optimization for heterogeneous platforms, matching my interests and skillset perfectly. On the other hand, I believe that my experience with dependent type theory and functional programming, along with my knowledge of the Rust programming language, make the "Dependently-typed Functional Programming for Machine Learning" project an appealing choice. More on the systems side, I believe I can leverage my knowledge of operating systems and computer architecture to take part in the "Enabling Efficient Disaggregated Serves in Data Centers" project.

I believe that I can make a meaningful contribution to these research projects if selected. I hope to take advantage of this opportunity to obtain invaluable research experience, connections, and knowledge, which I hope to apply in my career as a systems researcher.

\noindent
Thank you very much for your time and consideration,

\vspace{5mm}
\noindent
Sincerely,

\noindent
Qingyuan Qie
\end{document}